\documentclass{article}
\usepackage{amssymb, amsmath, amsfonts, amsthm, graphics, enumerate, mathrsfs, mathtools, tikz-cd}
\usepackage{geometry, soul, hyperref}
\usetikzlibrary{positioning,arrows,scopes}


\usepackage{fancyhdr}
\pagestyle{fancy}

% \setlength{\headheight}{26pt}
% \setlength{\oddsidemargin}{-0.2in}    
% \setlength{\evensidemargin}{-0.2in}
% \setlength{\topmargin}{0in}
% \setlength{\textwidth}{6.5in}
 %\setlength{\headwidth}{6.5in} 
 \setlength{\textheight}{8.5in}
 
\theoremstyle{definition}
\newtheorem*{thm}{Theorem}
\newtheorem*{pthm}{Pre-Theorem}
\newtheorem{prob}{Problem}
\newtheorem{ex}{Exercise}
\newtheorem*{dfn}{Definition}
\newtheorem{eg}{Example}
\newtheorem*{prop}{Proposition}
\newtheorem*{rmk}{Remark}
\newtheorem*{conj}{Conjecture}

\newcommand{\CC}{\mathbb{C}}
\newcommand{\FF}{\mathbb{F}}
\newcommand{\RR}{\mathbb{R}}
\newcommand{\ZZ}{\mathbb{Z}}
\newcommand{\QQ}{\mathbb{Q}}
\newcommand{\PP}{\mathbb{P}}
\newcommand{\cO}{\mathcal{O}}
\newcommand{\cU}{\mathcal{U}}
\newcommand{\cM}{\mathcal{M}}

\DeclareMathOperator{\Sym}{Sym}
\DeclareMathOperator{\Div}{Div}
\DeclareMathOperator{\Supp}{Supp}
\DeclareMathOperator{\indeg}{indeg}
\DeclareMathOperator{\Pic}{Pic}
\DeclareMathOperator{\Hom}{Hom}
\DeclareMathOperator{\im}{im}

\newcommand{\dsp}{\displaystyle}
\newcommand{\floor}[1]{\lfloor{{#1}}\rfloor}
\newcommand{\ceil}[1]{\lceil{{#1}}\rceil}

\title{log(m) meeting 10}
%\date{\today}
\lhead{ Log(M): Origami on lattices \\}
\rhead{ Meeting 10 \\ 29 March 2019}
\begin{document}


\section{Logistics}

\ul{Next meeting}:  Tuesday April 2, 11:30 - 1pm
\medskip

    
\bigskip
{Expectations} for next meeting: 
\begin{itemize}

    
    
    \item mountain / valley labellings
    for hexagon crease pattern
    
    \textbf{Problem 9}: 
    In a diagram with $6$ creases coming together at a vertex at equal angles,
the unit null cone near the flat space is two disjoint spheres.
We consider cutting up  these spheres into regions according to mountain/valley labellings.
\begin{enumerate}[(a)]
    % \item How many zero-dimensional, one-dimensional, and two-dimensional regions are there on each sphere?
    % (Answer: $9$, $24$, and $17$)
    
    \item[(b)] What shapes are the two-dimensional regions? 
    How do they fit together? 
    
    ($\rightarrow$ draw the diagrams we made in this meeting)
\end{enumerate}
    
   
    \item angle constraints on hexagon crease pattern
    
    \textbf{Problem 11}:
    Suppose we want to fix three consecutive creases on a hexagon at fold angles
    $\theta_1$, $\theta_2$, and $\theta_3$. 
    (By ``fold angle'' we mean flat $\Leftrightarrow \, \theta_i=0$.)
    \begin{enumerate}[(a)]
    \item What combinations of $\theta_1,\theta_2,\theta_3$
    are possible for a hexagon?
    Find exact equations to characterize this. 
        (Hint: it may help to use \url{https://en.wikipedia.org/wiki/Rodrigues\%27_rotation_formula})
    
    For example, if $\theta_2 = 0$ and $\theta_1, \theta_3 > 0$ then this is possible (and we can compute $\theta_4,\theta_5,\theta_6)$,
    but if $\theta_2=0$, $\theta_1 >0$, $\theta_3<0$ 
    then this is not possible for the hexagon.
    
    % \item When this is possible, what are the values of angles $\theta_4,\, \theta_5,\, \theta_6$?
    % Note: there may be more than one answer. How many answers are possible?
    \end{enumerate}
    Particular sub-problem:  
    
    \textbf{Problem 11 (c)}
     Suppose $\theta_1 = \theta_3 = 0$ and $\theta_2$ is arbitrary.
    Then one possibility is that $\theta_4 = \theta_6 = 0$
    and $\theta_5 = \theta_2$.
    A second possibility is that crease number $5$ is ``flipped inward'',
    so that $\theta_5 = -\theta_2$, and $\theta_4$ and $\theta_6$
    are at some positive angle. 
    What is the angle of $\theta_4= \theta_6$ as a function of $\theta_2$, in this case?
    
    Sub-sub-problem: 
    
    \textbf{Problem 11 (d)}
    Consider a spherical rhombus, 
    i.e. a quadrilateral on a unit sphere with all four side lengths $L$.
    Show that opposite pairs of angles are equal, say $\alpha$ and $\beta$.
    \begin{center}
        \includegraphics[scale=0.25]{rhombus}
    \end{center}
    Show that $\alpha$ and $\beta$ satisfy the relation
    \[ \tan \frac\alpha{2} \tan \frac\beta{2} = \frac1{\cos L} .\]
    Graph this relation on $\alpha$, $\beta$ for $L = \displaystyle\frac\pi{3}$
    (i.e. the spherical length of a hexagon edge).
    
    (Check: As $L \to 0$, why does this agree with the relation 
    $\alpha + \beta = \pi$
    for flat geometry?)
    
    \item \textbf{[Writing]} 
    Write up notes for this meeting, and
    continue writing up relevant discussion from this week in draft of final report 
    \item \textbf{[Final project]}
    Think about whether you would rather create a final report in article form,
    or a hexagon visualization program as a final project, or both 
    (in addition to the poster)

\end{itemize}




\end{document}